\section{Lessons Learned Perspective}

Describe the biggest issues, how you solved them, and which are major lessons learned with regards to:
\begin{itemize}
  \item Evolution and refactoring
  \item Operation
  \item Maintenance
  \item Of your ITU-MiniTwit systems. Link back to respective commit messages, issues, tickets, etc. to illustrate these.
  \item Also reflect and describe what was the "DevOps" style of your work. For example, what did you do differently to previous development projects and how did it work?

\end{itemize}

\subsubsection{DevOps way of working}

Throughout the project, we as a group have kept the “three ways” characterizing DevOps in mind; flow, feedback and continual learning and experimentation (REFERENCE TO THE DEVOPS HANDBOOK). As deployment have not been a part of our former courses, the principles of flow do not deem so much relevant comparison as the other principles.

One of the main differences from other group work is clearly centered around the feedback principle. In earlier projects, we developed tests as a part of the development, but they were rarely executed. This resulted in hours of development, where the newly developed code was not tested. Therefore, we often had to scrap code and rewrite it. During this project, after the setup of our pipelines, we have been able continuously test our project. We have also used tools such as Sonarcloud and CodeQL. This has led us to not only develop better code, but also ensured that the code was functional and “safe” before it was deployed. This one of the aspects the DevOps way of working impacted our work the most, as it was a huge improvement from how we formerly have been developing.