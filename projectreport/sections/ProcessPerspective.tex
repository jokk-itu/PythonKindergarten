\section{Process' perspective}

\textbf{Interactions as developers}\newline
  %How do you interact as developers?
  Due to the current Covid situation, the team has been forced to meet and work online. The team typically meets on Discord in the DevOps
  exercise sessions, during the week, and in the weekends. Often, collaboration is done through Visual Studio Code's Live Share (which enables group/pair programming) or working through the tasks/exercises individually, helping each other along the way. 
  \newline
 
%\newline
\textbf{Team organization}\newline
  %How is the team organized?
  The team strives to work in an agile manner, proceeding forward step by step, and adjusting work flows to meet ongoing challenges. Mainly, the group will work together as one big group. 
  However, occasionally, the group splits into two to increase effectiveness, whereby each group focuses on a selected topic. Both groups will then reconvene to demonstrate their findings. Remaining work is distributed to the individual group members, assisting each other in the process.

  
  \begin{itemize}
  \item A complete description of stages and tools included in the CI/CD chains.(That is, including deployment and release of your systems.)
  \item Organization of your repositor(ies).(That is, either the structure of of mono-repository or organization of artifacts across repositories. --> In essence, it has to be be clear what is stored where and why.) 
  %Applied branching strategy.
  \item For development purposes, the team uses Git for Distributed Version Control and have adopted a long-running branches approach to branching. In this context, there are two main branches, master and develop 
  (i.e. we have called it 'development' branch) as well as many short-lived topic branches, which are used for implementing features. This approach to branching works well in a centralized workflow such as ours, where the project is collaborated on in a shared
  repository (i.e. Github).
  %Applied development process and tools supporting it (For example, how did you use issues, Kanban boards, etc. to organize open tasks)
  \item The Projects Board on Github were used to categorize open tasks and organize our development work. Specifically, the team would create an issue with a description added to the comments section along
  with attached labels, such as 'need to have', 'group discussion', etc. These same issues were then assigned to different group members and became the tasks that needed attention on the Projects Board.
  
  \item How do you monitor your systems and what precisely do you monitor?
  \item What do you log in your systems and how do you aggregate logs?
  \item Brief results of the security assessment.
  \item Applied strategy for scaling and load balancing.
  \item In essence it has to be clear how code or other artifacts come from idea into the running system and everything that happens on the way.
  
\end{itemize}