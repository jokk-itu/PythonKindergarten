\section{Process' perspective}

\textbf{Interactions as developers}\newline
 How do you interact as developers?\newline
 
\newline
\textbf{Team organization}\newline
  How is the team organized?\newline
  
  \newline
  \textbf{A complete description of stages and tools included in the CI/CD chains.(That is, including deployment and release of your systems.)}\newline
  The tools we used for CI/CD chains were Travis and GitHub Actions. We started working with Travis, which worked very well for us, unfortunately we experienced some problems with expenses in Travis. We decided to migrate to GitHub Actions, due to this.
  We migrated to GitHub Actions, and this works just as well as Travis for us. GitHub Actions is an easy tool to use directly from GitHub. Since we are using GitHub to store our repository, GitHub Actions is a good choice since, it is easy to integrate, easy to duplicate workflow, works faster, and has a better ecosystem of actions in a centralised app/action store.. It has also made it easy for observing the status of the pipelines, since it is all on GitHub. 
  Our stages include build, deploy and test. When we deploy we have the following stages, \textit{building and testing}, \textit{CodeQL} that autobuild attempts to build any compiled languages, \textit{Sonnarscanner} that runs project analysis, \textit{DockerBuild} which build the project, then \textit{deploying} and finally \textit{releasing} it all.\newline
 
  \newline
  \textbf{Organization of your repositor(ies).(That is, either the structure of of mono-repository or organization of artifacts across repositories. In essence, it has to be be clear what is stored where and why.)} \newline
  For our project we have chosen to use the structure of mono-repository. The reason for choosing this structure, was that during this project, we were only building one system. Therefor we thought it would be best to keep everything in the same repository, that goes by the name PythonKindergarten. \newline
  
  \newline
  \textbf{What do you log in your systems and how do you aggregate logs?}  \newline
  In our system we are logging to ElasticSearch from our API, and it uses SeriLogs to send these logs. 
  We are logging our simulation errors, and have divided them into errors regarding, follow, tweet, unfollow, connectionError, readTimeout and Register.
  To aggregate these logs, we are using ElasticSearch and Kibana. We use ElasticSearch to store our logs in dedicated log indexes, and Kibana is used as a visualization tool for these logs in ElasticSearch. This makes it easy to keep track of our system in and quickly discover if there is anything wrong. \newline
  
  \begin{itemize}

  \item Applied branching strategy.
  \item Applied development process and tools supporting it (For example, how did you use issues, Kanban boards, etc. to organize open tasks)
  \item How do you monitor your systems and what precisely do you monitor?
  \item Brief results of the security assessment.
  \item Applied strategy for scaling and load balancing.
  \item In essence it has to be clear how code or other artifacts come from idea into the running system and everything that happens on the way.
  
\end{itemize}