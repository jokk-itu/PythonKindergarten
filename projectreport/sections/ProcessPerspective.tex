\section{Process' perspective}

\textbf{Interactions as developers}\newline
 How do you interact as developers?\newline
 
\newline
\textbf{Team organization}\newline
  How is the team organized?\newline
  
  \newline
  \textbf{A complete description of stages and tools included in the CI/CD chains.(That is, including deployment and release of your systems.)}\newline
  The tools we used for CI/CD chains were Travis and GitHub Actions. We started working with Travis, which worked very well for us, unfortunately we experienced some problems with expenses in Travis. We decided to migrate to Github Actions, due to this.
  We migrated to GitHub Actions, and this workes just as well as Travis for us. GitHub Actions is an easy tool to use directly from GitHub. 
  Our stages include build, deploy and test. When we deploy we have the following stages, \textit{building and testing}, \textit{CodeQL} that autobuild attempts to build any compiled languages, \textit{Sonnarscanner} that runs project analysis, \textit{DockerBuild} which build the project, then \textit{deploying} and finally \textit{releasing} it all.\newline

  \newline
  \textbf{Organization of your repositor(ies).(That is, either the structure of of mono-repository or organization of artifacts across repositories. In essence, it has to be be clear what is stored where and why.)} \newline
  For our project we have chosen to use the structure of mono-repository. The reason for choosing this structure, was that during this project, we were only building one system. Therefor we thought it would be best to keep everything in the same repository, that goes by the name PythonKindergarten. \newline
  
  \newline
  \textbf{What do you log in your systems and how do you aggregate logs?}  \newline
  In our system we are logging all the dependencies that we are using, a few of these are Docker, Grafana and SonarCloud. 
  We are also logging our simulation errors, and have divided them into errors regarding, follow, tweet, unfollow, connectionError, readTimeout and Register.
  To aggregate these logs, we are using ElasticSearch and Kibana. We use ElasticSearch to store our logs in dedicated log indexes, and Kibana is used as a visualization tool for these logs in ElasticSearch. This makes it easy to keep track of our system in and quickly discover if there is anything wrong. (See apendix __) \newline

  
  \begin{itemize}

  \item Applied branching strategy.
  \item Applied development process and tools supporting it (For example, how did you use issues, Kanban boards, etc. to organize open tasks)
  \item How do you monitor your systems and what precisely do you monitor?
  \item Brief results of the security assessment.
        The identified sources are our web services, for logging, monitoring and our MiniTwit application.
        The servers we use to host are also listed, as well as Docker and Nginx.
        The threat sources are XCSS, our firewall (UFW), Docker ignoring UFW and a DDoS attack.
        
  \item Applied strategy for scaling and load balancing.
  \item In essence it has to be clear how code or other artifacts come from idea into the running system and everything that happens on the way.
        Ideas starts of as issues or ideas at group meetings.
        Then a developer is assigned and checks out from the development branch,
        and thus creates a topic branch for the feature.
        When the feature is finished being implemented, 
        two other developers are assigned to look through
        the code and when approved, it is merged into development.
        When other features, which form a bigger feature, are all finished and merged into development.
        Then the development branch is merged into the master branch.
        Our CI and CD pipelines are then run, and will fail 
        If the project cannot build nor the test suite being run.
        It the pipelines succeed, the code will be pulled from 
        the main server in our docker swarm.
        Then the main server, will notify its' worker servers,
        to update their containers of the running application,
        with the new docker image. The docker image is pulled from dockerhub.
        
  
\end{itemize}